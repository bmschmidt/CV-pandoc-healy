\documentclass[11pt, letter]{article}

% This is a tex file to make Ben Schmidt's CV.
% The content is 

\usepackage{jk-vita}
\usepackage[notes,natbib,isbn=false,backend=biber,url=false,numbermonth=true]{biblatex-chicago}

%\usepackage{biblatex}

% Your biblatex file is likely somewhere else.

\title{}
\name{$name$}
% I guess postnoms is like junior? I dunno.
\postnoms{}
\address{
$for(address)$
  $address$\\
$endfor$
}
\www{$www$}
\email{$email$}
\tel{$tel$}
\twitter{$twitter$}
\subject{}


\begin{document}

\maketitle




\section{appointments}

$for(appointment)$
\subsection{$appointment.place$}
$for(appointment.items)$
$appointment.items.item$. $appointment.items.date$. \\
$endfor$
$endfor$

\section{education}
% Education uses a different field called ``items.''
$for(education)$
\subsection{$education.place$, $education.item$, $education.date$}
$if(education.info)$
$for(education.info)$
$if(education.info.text)$
% Store certain things in a field called 'text' to allow another field
% for priority.
$education.info.text$ \\
$else$
$education.info$ \\
$endif$
$endfor$
$endif$
$endfor$

% A little wonky. I want my publications divided up based on tags.
% (Yes, it would be better to just do this with bib file). That's for later.

\section{publications}

\subsection{Academic Publications}

$for(publication)$
$if(publication.academic)$
$if(publication.citekey)$
\cite{$publication.citekey$}.\\[.15cm]
$else$
$if(publication.author)$$publication.author$,$endif$
``$publication.title$.'' $if(publication.journal)$\textit{$publication.journal$}$if(publication.journal_info)$, $publication.journal_info$$endif$.$endif$ $publication.date$$if(publication.description)$. $publication.description$$endif$.\\[.15cm]
% End of block for academic publications
$endif$ 
$else$

$if(publication.public)$
$else$
% Raise an error if some publication is neither public nor academic;
% otherwise I'll lose some out of stupidity.
\error{$publication.title$}
$endif$
$endif$
$endfor$

\subsection{General Audience Publications}

$for(publication)$
$if(publication.public)$
$if(publication.citekey)$
\cite{$publication.citekey$}.\\[.15cm]
$else$
$if(publication.author)$$publication.author$,$endif$
``$publication.title$.'' $if(publication.journal)$\textit{$publication.journal$}$if(publication.journal_info)$, $publication.journal_info$$endif$.$endif$ $publication.date$$if(publication.description)$. $publication.description$$endif$.\\[.15cm]
$endif$
% End of block for general publications
$endif$
$endfor$

\section{grants and fellowships}

$for(grants)$
$grants$\\[.15cm]
$endfor$

\section{invited talks}

$for(invited_talk)$
``$invited_talk.title$.'' $if(invited_talk.host)$$invited_talk.host$, $endif$$invited_talk.place$. $invited_talk.date$.\\[.15cm]
$endfor$


\section{workshops}

$for(workshop)$
``$workshop.title$.''
$if(workshop.host)$$workshop.host$, $endif$$workshop.place$.
$workshop.date$.\\[.15cm]
$endfor$

\section{conference talks}

$for(conference)$
``$conference.title$.''
$if(conference.host)$$conference.host$, $endif$
$conference.place$.
$conference.date$.\\[.15cm]
$endfor$

\section{courses taught}

$for(teaching)$
``$teaching.title$.'' $teaching.date$. ($teaching.type$)\\[.15cm]
$endfor$


\section{public history}

$for(public_history)$
% This one has a hierarchical setup: there are multiple roles in public history, and
% each can have any number of ``gigs''
% These are irregular enough that I just use an ``item'' field to store the text.
\subsection{$public_history.role$}
$for(public_history.gigs)$
$public_history.gigs.item$ $public_history.gigs.date$.\\[.15cm]
$endfor$
$endfor$
\subsection{Selected media coverage}

$for(media_coverage)$
$if(media_coverage.citekey)$
\fullcite{$media_coverage.citekey$}\\
$else$
$media_coverage.title$
$endif$
$endfor$


\section{service}
$for(service)$
\subsection{$service.type$}
$for(service.gigs)$
$service.gigs.item$. $service.gigs.date$.\\[.15cm]
$endfor$
$endfor$


\section{research competencies}
$for(competencies)$
\subsection{$competencies.type$}
$for(competencies.items)$
$competencies.items$\\[.15cm]
$endfor$
$endfor$



\end{document}
